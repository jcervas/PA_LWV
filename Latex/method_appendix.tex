% To assist us in our evaluations of evidenced for partisan gerrymandering in these plans we also consider, for comparison purposes, four other maps drawn conforming either entirely or almost entirely to what the Pennsylvania Supreme Court referred to as traditional districting criteria (often referred to as ``good government" criteria). One is taken from an on-line blog the \textit{DailyKos} (\href{https://bit.ly/2o3PeuX}{Brief in Support of Proposed Remedial Districting Plans as Amici Curiae Adele Schneider and Stephen Wolf, LWV, PA. No. 159 MM 2017}); the other three were drawn by the present authors. Three of the four maps were drawn with no election outcome or party registration data used in their creation.
\par
%The criteria we examine in evaluating plans are only those highlighted by the Pennsylvania Supreme Court or found in the expert witness testimony in the case, such as the \textit{mean/median gap} \citep{Best2018} and the \textit{efficiency gap} \citep{McGhee2017_ELJ}. \new{Thus, we do not provide evidence about partisan bias as defined in \citet{Tufte1973}, though this was once thought to be the most promising approach to detecting and measuring gerrymandering \citep{Grofman2007}, or more recent measures introduced, such as \textit{declination} \citep{Warrington2018} in the main text, though we provide this data in the Appendix}. 
% We do not discuss district-specific criteria for identifying gerrymandering, since, although some district-specific evidence was presented about the 2011 map in the trial testimony by an expert who was highly knowledgeable about communities of interest in the state, the \textit{LWV} opinion itself offer a statewide standard for a finding of unconstitutional partisan gerrymandering. Consequently, we limit ourselves metrics that report results for a jurisdiction as whole.
% ================================================================= 
% -- FOOTNOTE -- FOOTNOTE -- FOOTNOTE -- FOOTNOTE -- FOOTNOTE --  %
% -----------------------------------------------------------------
        % \footnote{Given the U.S. Supreme Court's remand of the \textit{Gill} case, it appears that any federal standard would need to be district specific. \new{The court heard oral arguments in two partisan gerrymandering cases on March 26, 2018 (Rucho, et. al. v. Common Cause, et. al [1:16-CV-1026]) \& (Lamone, et. al. v. Benisek, et. al. [1:13-cv-03233-JKB]), with decisions expected in June 2019. In oral arguments, Justices Gorsuch and Kavanaugh directed their questions asking if measures of partisan gerrymandering are really an attempt at creating proportional representation, which the court previously stated is not guaranteed by the US Constitution in \textit{Davis v. Bandemer} (1986). In fact, unlike in \textit{Gill}, the words \textit{efficiency gap} were uttered only once in \textit{Rucho} and not at all in \textit{Lamone}, by Justice Gorsuch in a comment comparing the proportionality claims being made in the current case essentially being the same as those in \textit{Gill}.}}
% ----------------------------------------------------------------- 
% -- END FOOTNOTE -- END FOOTNOTE -- END FOOTNOTE -- END FOOTNOTE %
% =================================================================
\par
% In our discussion of how to judge whether a plan as a whole is a partisan gerrymander, we highlight an important distinction between \textit{neutral plans} and \textit{fair plans} -- each of which reflects a different approach to defining what constitutes gerrymandering. \textit{Neutral plans} refer to those that are drawn entirely with respect to traditional good government criteria with no attention paid to partisan considerations. One way to define partisan gerrymandering is with respect to a baseline defined by the set of feasible \textit{neutral plans} -– and this was how one of the plaintiff’s experts in the case, Professor Jowei Chen \cite[cf. ][]{Chen2013} shaped his testimony.
% % ================================================================= 
% % -- FOOTNOTE -- FOOTNOTE -- FOOTNOTE -- FOOTNOTE -- FOOTNOTE --  %
% % -----------------------------------------------------------------
%         \footnote{Similarly, \citet{Grofman2018_ELJ} asserts that ``Gerrymandering occurs when a districting plan creates a disparate treatment of the vote share of the minority and majority voting blocs in a way that penalizes the minority in its ability to translate its voting support into seats compared to what we might expect from a plan drawn on the basis of neutral principles."}
% % ----------------------------------------------------------------- 
% % -- END FOOTNOTE -- END FOOTNOTE -- END FOOTNOTE -- END FOOTNOTE %
% % =================================================================
% In contrast, as we use the term, \textit{fair plans} are ones that recognize the partisan realities of electoral geography, and seek to limit the expected disproportionality between a party’s vote share and its seat share.
% % ================================================================= 
% % -- FOOTNOTE -- FOOTNOTE -- FOOTNOTE -- FOOTNOTE -- FOOTNOTE --  %
% % -----------------------------------------------------------------
%         \footnote{There have been two approaches to measuring ``fairness" in the academic literature on gerrymandering. The best known is based on the idea of \textit{partisan asymmetry}, i.e., equals being treated unequally. For two-party competition, the intuition underlying this approach is that, if a given vote share for party A will result in a different seat share for that party than what party B would receive with an identical vote share, this can be taken to be evidence of \textit{partisan bias}.  \textit{Partisan bias} can be calculated from a (hypothetical) seats-votes curve, with the \textit{locus classicus} \citet{Tufte1973}, and important recent work done by \citet{GelmanKing1994_unifiedAJPS} among others \citep[see review in][cf. \citealt{Brookes1959, Brookes1960, Johnston2002}]{Grofman2007}, as implemented with some improvements by scholars such as \citet{Johnston1994, Rossiter1997}, although the basic idea goes back at least as far as \citet{Dahl1956}. 
%         \par 
%         A second approach to fairness is based on having some reasonable relationship between the proportion (or expected proportion) of congressional or legislative seats captured by a given party and that party’s share (or expected share) of the \underline{statewide} two-party vote -- with proportionality of vote share and seat share the obvious ideal.  However, the U.S. Supreme Court has repeatedly rejected the claim that any group has a right to proportional representation (\textit{Davis v. Bandemer}, 1986). The \textit{efficiency gap}, which compares the aggregate cracking and packing of each of the two major party’s voting strength \citep{McGhee2017_ELJ, Mcghee2018_Rejoinder_ELJ}, and was proposed as a measure of partisan gerrymandering in testimony by Professor Christopher Warshaw at the trial phase of the \textit{LWV} litigation, and \new{by others} in other recent litigation at the federal level, can be viewed as a fairness-based approach to gerrymandering. But it does not have perfect proportionality as its ideal. As its inventor, \citet[][pg. 69]{McGhee2014_Bias_LSQ}, \citep[see also][pg. 6]{McGhee2017_ELJ}, has pointed out, the \textit{efficiency gap} shows the absence of gerrymandering only when the gap between a party’s seat share and 50 percent is twice the difference between its vote share and 50 percent. Higher values of the \textit{efficiency gap} are taken as undesirable/evidence of partisan gerrymandering. 
%         \linebreak
%         \par
        
%         \bg{Should we exclude this per reviewers saying not necessary to pick fights?}
%         \new{
%       These two approaches to fairness can yield quite different conclusion about gerrymandering, and we believe only the first is appropriate, especially since the literature on the \textit{efficiency gap} provides no justification for why a value of two is the ideal. Our view is that disproportionality \textit{per se} is not an indicator of gerrymandering, since it is well recognized that, in plurality single member district elections, the smaller party will be disadvantaged \citep{Gudgin2012}. Rather, gerrymandering is best tapped by measures of disparate partisan impact such as partisan bias a la \citet{Tufte1973}, or the \textit{mean-median gap} proposed by \citet{Best2018}, or other asymmetry measures, such as those proposed by \citet{Wang2016_SLR, Wang2016_ELJ}. The \textit{efficiency gap}, in contrast, is a very close relative of what \citet{Tufte1973} refers to as \textit{responsiveness} (swing ratio).  Of course, as we discuss below in the context of Pennsylvania congressional districting, even if we find evidence of partisan gerrymandering, we must ask whether it can be explained by either chance factors or the partisan electoral geography of the jurisdiction rather than intentional gerrymandering, and courts must decide when a given level of demonstrated gerrymandering rises to the level of a constitutional violation.}
%       }
% % ----------------------------------------------------------------- 
% % -- END FOOTNOTE -- END FOOTNOTE -- END FOOTNOTE -- END FOOTNOTE %
% % =================================================================
% In so doing, they will seek to compensate for any biases that may be imposed because one party’s voting strength is more geographically concentrated than another’s. When gerrymandering is judged from the standpoint of political \textit{fairness} rather than \textit{neutrality}, such geographically-linked biases can create what has been called a ``natural gerrymander."
% \par
% In Pennsylvania, for example, Philadelphia is overwhelmingly Democratic in voting. In particular, if you draw two congressional districts entirely within Philadelphia County, one of them is very likely to give Democratic candidates around 90\% of the vote, and for sure, as long as both are wholly within the County, the average vote in the two will be around 80\% Democratic no matter how you draw the two districts. There are no other large, concentrated pockets of equivalently overwhelmingly Republican voting strength. Thus, more Democratic votes will ``naturally" be ``wasted" in Philadelphia than Republican votes will be ``wasted" elsewhere in the state. \new{Additionally, the city of Philadelphia is landlocked, ie, it's position on the border reduces the ability to effectively distribute these voters among many districts in ways to create \textit{fair} plans.} Additional Democratic wasted votes come in heavily Democratic Allegheny County, the county in which the city of Pittsburgh is located. Such geographic features need to be taken into account in assessing the extent of intentional partisan gerrymandering, and the expert witness testimony in the case by experts for plaintiffs by and large did so.
% \par
% Because electoral geography can matter, distinguishing purposive effects of line drawing from consequences of ``natural gerrymandering" can be critical for courts that are seeking to assess partisan intent indirectly by looking at partisan consequences and positing that what is achieved/expected, and foreseeable by any reasonable person, can be seen as intended. But even if partisan \underline{intent} is not at issue, a court that is paying attention to partisan \underline{consequences} when it is drawing a remedy plan might have to decide to what extent it is seeking to draw a ``neutral" plan as opposed to drawing a ``fair" plan as its remedy -- using the definition of those terms given above.
% \par
% In the discussion that follows in the next section, even after taking the partisan geography into account, we argue that the \textit{2011 Enacted map} was a blatant and undisguised pro-Republican gerrymander, while the 2018 \textit{Joint legislative} map proposed by Republican legislators was a covert or \textit{stealth gerrymander}. However, notwithstanding claims to the contrary, just as we would expect, the 2018 court drawn \textit{Remedial} map was not a gerrymander. Furthermore and rather unexpectedly, the plan proposed by the \textit{Gov. Wolf}(D) also could not be labeled as a pro-Democratic partisan gerrymander.
% 
% \par \bg{We should formally define what a stealth gerrymander is and how we measure it, for future citations. Here's my first crack.} \par
% \new{\textit{Stealth Gerrymander} - A plan that is indistinguishable from ``neutral" plans in terms of traditional redistricting criteria, but non-the-less creates a partisan advantage for one party. We can test for a \textit{stealth gerrymander} by comparing plan statistics; to do so, a counterfactual set of districts must be established. These counterfactual plans must also follow traditional districting criteria, especially measures of compactness and splits of political subdivisions. Partisan bias can then be tested by simulating a seats/votes curve and applying a regression, such as the one described in \citet{Tufte1973}. This test will measure both the partisan bias in the plan, along with it's responsiveness to changes in vote shares. A \textit{Stealth Gerrymander} will return statistics on traditional districting criteria that are indistinguishable from the set of counterfactual districts, but have partisan bias and responsiveness numbers that lie on the exteriors of the range.}

% ================================================================= 
% -- FOOTNOTE -- FOOTNOTE -- FOOTNOTE -- FOOTNOTE -- FOOTNOTE --  %
% -----------------------------------------------------------------
%        \footnote{As noted above, there is dispute in the academic literature about what elections to use in making projections, and in what other factors, if any, to include in the projections \citep{Rush2000}. Nonetheless, there can be no dispute that those doing the redistricting make use of past election data to make judgments about expected partisan outcomes and commonly do so with considerable confidence. Our view is that there is no \textit{a priori} way to determine optimal projection methods prior to actually running analyses of their predictive accuracy. \new{Our composite measure has a mean discrepancy of 3.4\% after removing noncompetitive elections, and predicts every district accurately. There is also a 94\% correlation between the projected outcomes and the actual outcomes, after adjusting for noncompetitive districts.} We also recognize that projection methods that appear deterministic are nonetheless probabilistic in their predictions, even if that point is not made \underline{explicitly}. Moreover, for most plans, we would expect to get different outcomes in very good years for Republicans than in very good years for Democrats. What makes for a durable gerrymander is the absence of competitive districts \citep{Owen1988}. This absence operates to constrain the responsiveness of the plan to changes in voter preferences and puts limits on the possible success of the minority party in other than tsunami–like wave elections. \new{Figure \ref{fig:densityplots_congressional} explicitly demonstrates this.} For alternative perspectives see e.g., \citet{Mcghee2018_Rejoinder_ELJ}, \citet[][esp. at p. 83]{Best2018}.}
% ----------------------------------------------------------------- 
% -- END FOOTNOTE -- END FOOTNOTE -- END FOOTNOTE -- END FOOTNOTE %
% =================================================================
%
\new{While we would like to test the accuracy of the measure for 2018 result, a number of circumstances make direct comparison unavailable. Voter Tabulation Districts (VTDs) constant from election to election, and in particular they would differ from 2016 since the congressional districts themselves have changed, requiring subsequent VTD changes. Second, the district names have changed for the court plan, making direct comparison impossible. Still, we can identify the \textit{predicted} number of districts in 2018, understanding it as a probabilistic estimation. As Table \ref{tab:plansummary} shows, our composite measure averages 51.4\% of the seats (9.26R-8.74D seats at 48.5\% of the vote), once we account for the difference in state-wide support the parties actually received in 2018. Nearly a quarter of the simulations resulted in the actual outcome of 9R-9D, and 65.6\% were within one seat.} 
% ================================================================= 
% -- FOOTNOTE -- FOOTNOTE -- FOOTNOTE -- FOOTNOTE -- FOOTNOTE --  %
% -----------------------------------------------------------------
    \footnote{\new{We can characterize the overall validity of our proxy as strong. Of the three observable elections, only 2012 can be classified as a miss. 2012 was the first election after redistricting, and changes in incumbency and other factors might be contributing to the slight miss, along with any electorate shifts or effects of candidates for the Presidency. Additionally, partisan bias seemed to be particularly large given that the Democrats recieved over 50\% of the aggregated votes, but were elected in only 5 of the 18 districts. Since we are only interested in the effects of the plan post-2016, we are confident that the composite measure is appropriate for measuring gerrymandering under different plans.}}
% ----------------------------------------------------------------- 
% -- END FOOTNOTE -- END FOOTNOTE -- END FOOTNOTE -- END FOOTNOTE %
% =================================================================


% ================================================================= 
% -- FOOTNOTE -- FOOTNOTE -- FOOTNOTE -- FOOTNOTE -- FOOTNOTE --  %
% -----------------------------------------------------------------
        \footnote{ \href{https://bit.ly/2NsvXyv}{The precinct level data come from Harvard Dataverse for 2008}, and from \href{https://bit.ly/2PJMyzn}{Nathaniel Kelso’s github for 2016.}}
    % ------- END FOOTNOTE ----------------------------------------
% =================================================================
Using the 2016 composite of five state-wide elections, among the 18 districts in the Court map, five are strongly favored by the Democrats, seven are strongly favored by the Republicans, and six are competitive (\rpm 5\%).
% ================================================================= 
% -- FOOTNOTE -- FOOTNOTE -- FOOTNOTE -- FOOTNOTE -- FOOTNOTE --  %
% -----------------------------------------------------------------
        \footnote{For a discussion about thresholds of competitiveness, see \citet{CervasGrofman2017_PC}. \new{For the partisan breakdowns, we utilize the actual sum of the five races instead of the simulated hypothetical elections we use to measure gerrymandering.}}
% ----------------------------------------------------------------- 
% -- END FOOTNOTE -- END FOOTNOTE -- END FOOTNOTE -- END FOOTNOTE %
% =================================================================
In the 2011 map, Democrats were strongly favored in only four districts, while Republicans were favored in eight. In the map adopted in 2011, the Republicans were able to consistently win 13 of the districts in every election, despite a statewide partisan split that straddled the 50\% line.
% ================================================================= 
% -- FOOTNOTE -- FOOTNOTE -- FOOTNOTE -- FOOTNOTE -- FOOTNOTE --  %
% -----------------------------------------------------------------
        \footnote{In 2012, the Republican share of the state-wide total popular vote in the congressional elections in Pennsylvania was 49.2\%, in 2014 it was 55.5\%, and in 2016 it was 54.1\%. In 2018, it was 44.9\%.}
% ----------------------------------------------------------------- 
% -- END FOOTNOTE -- END FOOTNOTE -- END FOOTNOTE -- END FOOTNOTE %
% =================================================================
%
% =================================================================
% ------- TABLE ---------------------------------------------------
    \begin{center}\textbf{[INSERT FIGURE \ref{fig:barplot_partisanship} ABOUT HERE]}\end{center}
                    
 
% =================================================================
% -- FIGURE -- FIGURE -- FIGURE -- FIGURE -- FIGURE -- FIGURE --   % 
% ----------------------------------------------------------------- 
\begin{figure}
    \begin{center}
    \caption{District Partisanship by Plan}
    \label{fig:barplot_partisanship}
    \includegraphics[width=1\textwidth]{Figures/fig_barplotPartisanship.pdf}
    \end{center}
    \hrulefill \
    \footnotesize{Note: All percentages in terms of Democratic share of the two-party vote from the composite measure of five state-wide races in 2016.}
\end{figure}
% -----------------------------------------------------------------
% -- END FIGURE -- END FIGURE -- END FIGURE -- END FIGURE -- END FI %
% ================================================================= 

% ------- END TABLE -----------------------------------------------
% =================================================================
\par
We show projections under this composite five election measure in Figure \ref{fig:barplot_partisanship}.  Here, and in Table \ref{tab:plan_estimates} in the Appendix, we provide more detail than in Table \ref{tab:plansummary} by ranking the districts from least Democratic to most Democratic in projected two-party vote share. We see that the Court plan, \textit{ceteris paribus}, is expected to yield a 9D-9R split \new{for the 2016 election}. However, because of its high number of competitive seats, this single number is somewhat misleading. Using the five-election model as our predictor, there are fewer safe Democratic seats (5) than safe Republican seats (7) in the Court plan, using a five-percentage point definition of a competitive seat. This is one fewer safe seat for Republicans and one additional safe Democratic seat than is found in the 2011 plan, but the Court plan also has five districts balanced on a knife edge, with a projected winner’s margin under two percentage points, making it highly responsive to changes in voting patterns. Thus, in the Court drawn plan, changes in vote choice or turnout
% ================================================================= 
% -- FOOTNOTE -- FOOTNOTE -- FOOTNOTE -- FOOTNOTE -- FOOTNOTE --  %
% -----------------------------------------------------------------
        \footnote{We do not take incumbency into account in our projections. If there is an incumbent in place, and if the district boundaries are not that much changed by redistricting, districts with highly competitive baseline partisan support will be biased toward the reelection of that incumbent. However, incumbency advantage has declined in recent years to lows which have not been seen since the 1950s \citep{Jacobson2015}. The 2016 general election average we used to calculate the projected votes include three state-wide races where the Democratic candidates did particularly well comparedd both to Hillary Clinton and to the Democratic candidate for U.S. Senate. Similarly, in Pennsylvania in 2016, Republican congressional incumbents out-polled President Trump in 9 of the 13 districts won by Republicans.}
% ----------------------------------------------------------------- 
% -- END FOOTNOTE -- END FOOTNOTE -- END FOOTNOTE -- END FOOTNOTE %
% =================================================================
might result in different outcomes, especially in the most marginal of districts. \new{The court plan, and those with similar relative district positions will likely see `all or nothing' seat share shifts, where the parties are limited in their ability to create disproportionate delegations.} Most importantly, changes in voter sentiment will result in incumbents losing their seats, \new{especially in those competitive seats,} precisely as envisioned by the founders.
\par
\new{The first draft of this essay was submitted before the 2018 election, so it is reasonable to ask how well our predictions did after an election was held in 2018. Indeed, as predicted in our model, the 2018 election, using the Court's Remedial map, result in a 9R-9D balance. Democrats had a good year, winning the popular vote by a 2.66 percentage point advantage. We would characterize a 9R-9D result as unbiased. We acknowledge that the Republicans won three districts by less than three percentage points, meaning the election could have easily been 6R-12D. Under the old map, measured by the composite 2016 data, it too predicts a 9R-9D split, but most tellingly, the average seat share is 52\% for the Republicans in the old map, compared to 47\% under the Remedial map. Moreover, the median share under the enacted map would have been 9R-9D, whereas under the Court Remedial plan, it was 8R-10D. At the time of submission, precinct level data was not available in a form that would facilitate evaluation of the five other maps analyzed in this essay.}
%

%%%%%%%%%%%%%%%%%%%%%%%%%%%%%%%%%%%%%%%%%%%%%%%%%%%%%%%%%%%%%%%%%%%%
% ================================================================ %
\subsubsection*{Evaluation of the Court Remedy in comparison with Republican and Democratic alternatives}
% ================================================================ %
%%%%%%%%%%%%%%%%%%%%%%%%%%%%%%%%%%%%%%%%%%%%%%%%%%%%%%%%%%%%%%%%%%%%
%
The \textit{Joint} Submission, which was \new{floated by the Republican legislature} but not passed by the legislature, was a collaboration of the Republican leaders of the two chambers, Rep. Turzai and Sen. Scarnati. While the plan adhered far more closely to the principles laid out by the court than the \textit{2011} map, the governor called it a partisan gerrymander and rejected it. He issued a statement saying ``\dots my preference would have been for the General Assembly to send me a fair map" and submitted his own map to the Court."
% ================================================================= 
% -- FOOTNOTE -- FOOTNOTE -- FOOTNOTE -- FOOTNOTE -- FOOTNOTE --  %
% -----------------------------------------------------------------
        \footnote{Official statement from the office of the Governor: \href{https://bit.ly/2CqIg88}{Governor Wolf Submits a Fairer Congressional Map to Supreme Court}}
% ----------------------------------------------------------------- 
% -- END FOOTNOTE -- END FOOTNOTE -- END FOOTNOTE -- END FOOTNOTE %
% =================================================================
The Court rejected both in favor of the plan drawn by the Court’s consultant.
\par
Using the 2016 five election composite projection, the \textit{Joint Legislative} plan would have delivered an average of seven Democratic seats, the same as the \textit{Governor}’s plan. However, there are very important differences between the two plans once one looks at the details. One main difference between the \textit{Joint} plan and the \textit{Governor}’s is that among the seven districts likely to be won by Democrats, the Democratic percentages are cut more narrowly in the Republican plan than in the \textit{Governor}’s plan in three districts so that, in a good Republican year, Republicans will do better under their plan than under the \textit{Governor}’s plan. The second main difference between the plans operates in the same direction. In the next four most competitive seats where a Republican is projected to win under our composite measure, the Democratic vote margins are lower under the \textit{Joint legislative} plan than under the \textit{Governor}’s plan. Thus, in a good Democratic year, the Democrats will likely not fare as well under the Republican plan as under the \textit{Governor}’s plan. In sum, whether it’s a good year for Republican or a bad year for Republicans, the Republicans will fare better under their own \textit{Joint} plan than under the \textit{Governor}’s plan.
%
%%%%%%%%%%%%%%%%%%%%%%%%%%%%%%%%%%%%%%%%%%%%%%%%%%%%%%%%%%%%%%%%%%%%
% ================================================================ %
\subsubsection*{Evaluation of the Court Remedy in Comparison with DailyKos Plan}
% ================================================================ %
%%%%%%%%%%%%%%%%%%%%%%%%%%%%%%%%%%%%%%%%%%%%%%%%%%%%%%%%%%%%%%%%%%%%
%
The liberal blog, the \textit{DailyKos}, also submitted a plan seeking to implement the criteria that the Court laid out.
% ================================================================= 
% -- FOOTNOTE -- FOOTNOTE -- FOOTNOTE -- FOOTNOTE -- FOOTNOTE --  %
% -----------------------------------------------------------------
        \footnote{DailyKos has prepared a number of other plans, but for space reasons we will only consider their non-partisan map. The DailyKos has long advocated for redistricting reform that creates good government plans that resemble proportional representation.}
% ----------------------------------------------------------------- 
% -- END FOOTNOTE -- END FOOTNOTE -- END FOOTNOTE -- END FOOTNOTE %
% =================================================================
Using our composite measure, as shown in Table \ref{tab:plan_estimates}, their plan created five safe Democratic districts, eight safe Republican districts, and five competitive districts. Based on the five-election model, in 2016 the Democrats would have been predicted to win eight seats and the Republicans 10 under this plan. Though this result is slightly worse for the Democrats in an absolute sense then the \textit{Court Remedy}, the ninth and tenth districts have margins so small that small permutations in votes could end up with an 8R-10D split. The Democrats might have preferred a plan like this over the \textit{Court's Remedy}. Like the other non-partisan plan analyzed here and those of the expert witnesses, the \textit{DailyKos} plans is highly responsive to public sentiment.
% ================================================================= 
% -- FOOTNOTE -- FOOTNOTE -- FOOTNOTE -- FOOTNOTE -- FOOTNOTE --  %
% -----------------------------------------------------------------
        \footnote{There are also illustrative maps proposed by academics, by various interest groups, and by journalists (such as \textit{FiveThirtyEight}) that we do not analyze due to space constraints and their similarity to other plans.}
% ----------------------------------------------------------------- 
% -- END FOOTNOTE -- END FOOTNOTE -- END FOOTNOTE -- END FOOTNOTE %
% =================================================================
%
%
%
%%%%%%%%%%%%%%%%%%%%%%%%%%%%%%%%%%%%%%%%%%%%%%%%%%%%%%%%%%%%%%%%%%%%
% ================================================================ %
\subsubsection*{Evaluation of the Court Remedy to Plans Drawn by the Authors}
% ================================================================ %
%%%%%%%%%%%%%%%%%%%%%%%%%%%%%%%%%%%%%%%%%%%%%%%%%%%%%%%%%%%%%%%%%%%%
%
We have drawn two alternative plans for the House using the good government criteria required by the court in which we paid absolutely no attention to partisan information, and a third plan that started with plan \textit{V1} as a base and attempted to increase the number of Democratic districts by switching out precincts and counties between districts using 2016 election results at the presidential level. All three plans score well on good government criteria. These plans have between five and six competitive seats. In a good Democratic year such as 2008, as shown in Table \ref{tab:plansummary}, the projected results in these plans would have been 7R-11D in Plan \textit{V1} and 8R-10D in Plans \textit{V2} \& \textit{V3} using presidential results. In a good Republican year such as 2016, we estimate the result would have been 11R-7D in Plan \textit{V1}, 10R-8D in Plan \textit{V2}, and 11R-7D in Plan \textit{V3}, using only presidential results. Using the 2016 five-election synthetic projection method, we find results of 10R-8D in Plan \textit{V1}, 11R -7D in Plan \textit{V2}, and 9R -9D in Plan \textit{V3}.  Comparing 2008 and 2016 projections, all three of our plans demonstrate responsiveness to shifting preferences of the electorate. As such, these plans, along with the \textit{DailyKos} and \textit{Governor's} plan can be classified as both \textit{neutral} and \textit{fair} plans.
%
%
% =================================================================
% -- SUBSECTION -- SUBSECTION -- SUBSECTION -- SUBSECTION -- SUBS %

\par
% =================================================================
% -- SUBSECTION -- SUBSECTION -- SUBSECTION -- SUBSECTION -- SUBS %
% =================================================================
    \subsection*{Results}
% =================================================================
% -- SUBSECTION -- SUBSECTION -- SUBSECTION -- SUBSECTION -- SUBS %
% =================================================================
% FOCUS RESULTS ON ENACTED, JOINT, AND REMEDIAL PLANS. OTHER 5 PLANS HAVE DATA IN TABLE


Table \ref{tab:gerry} in reports the measures of gerrymandering described above. No matter what measure one chooses to score the plans on, the \textit{2011 Enacted} unconstitutional plan is the worst, by far. The \textit{Joint Legislative} plan proposed by the Republican leaders of the state legislature is second worst on three of the four measures, and third worse on the other. By way of contrast, the \textit{Court Remedial} plan of 2018 scores the best on three of four measures, and second best on the other. The information provided in Table \ref{tab:gerry} show that bias, any way measured, always points in the direction of pro-Republican.  The magnitude of the bias varies by plan and by measure, though the unconstitutional \textit{2011 Enacted} plan stands out as especially egregious. \new{We cannot tell much about the process by which these districts were created or the intentions to gerrymandering based solely on any of the included measures. Though the unconstitutional plan of 2011 scores higher levels of bias then any other plan we have analyzed, it is difficult to derive meaning from these numbers since they don't include measures of uncertainty and the court has yet to establish a baseline, which when crossed, is \textit{indicia} of partisan gerrymandering. This type of analysis does allow use to measure the relative consequences of gerrymandering among plans, on an equal playing field, using identical electoral data. The results of these analyses provide additional evidence that the \textit{2011 Enacted} plans is without doubt an egregious gerrymander, the Joint legislative plans is a \textit{stealth gerrymander}, and the 2018 Court Remedial plan is \underline{not} a partisan gerrymander.} 

% We also offer an alternative way of presenting district specific vote share or projected vote share data, of the sort provided in Table \ref{tab:plan_estimates} of the main text. Figure \ref{fig:a1}’s shadings indicate the extent to which the districts are tilted toward one or the other party.    
% 
% 
% =================================================================
% -- SECTION -- SECTION -- SECTION -- SECTION -- SECTION -- SECTI %
% =================================================================
\section{Discussion about Partisan Gerrymandering in the Various Plans}
% •••••••••••••••••••••••••••••••••••••••••••••••••••••••••••••••••
What conclusions can we draw from the analyses in the previous section? Let us first turn to the \textit{2011 Enacted} plan. Here, there is simply no doubt that, using our standards and definition, it is a partisan gerrymander no matter what features of the map we examine. Of course, as the \textit{LWV} opinion emphasizes, this map demonstrate an almost unbelievable failure to satisfy traditional districting principles -- suggesting strongly the belief of its' authors that, given the U.S. Supreme Court’s history of abdication regarding partisan gerrymandering, there were no legal checks on how egregious a map could be drawn in search of partisan gain in the 2010 redistricting round \citep{McGann_et_al_2016_gerrymandering}. Indeed, it would be difficult to create a more egregious gerrymander. District 7 of the \textit{2011} map, otherwise known as ``Goofy kicking Donald Duck", best encapsulates this idea. The district was so specifically draw as to maximize Republican advantage that it narrows to just one road, and is splinter among so many municipalities that counting them by hand would be troublesome for even the most fluent mathematicians. The mere fact that the Democrats won over 50\% of the two-party vote in 2012 congressional elections but only won 27\% of the seats speaks to the efficiency of the gerrymander. Additionally, given that in the subsequent elections the two-party vote changed but the number of seats the Democrats won did not, the success of the plan is equally characterized by its unresponsive.
% ================================================================= 
% -- FOOTNOTE -- FOOTNOTE -- FOOTNOTE -- FOOTNOTE -- FOOTNOTE --  %
% -----------------------------------------------------------------
        \footnote{\citet{Owen1988} list two types of `optimal gerrymanders', in which we can classify the 2011 Pennsylvania plan as one that seeks to maximize expected number of seats, as opposed to one that seeks maximize the chances of a working majority.  The latter of the two is not likely to happen at the congressional district level, since, in most cases, the former actually can help lead to the latter when all 435 congressional districts are aggregated.}
% ----------------------------------------------------------------- 
% -- END FOOTNOTE -- END FOOTNOTE -- END FOOTNOTE -- END FOOTNOTE %
% =================================================================
\par
Let us now consider the Court adopted 2018 \textit{Remedial} plan. The antagonism of Republican state legislators to this map took on an extreme character. This included not just litigation to block the implementation of the map, but also threats to impeach the judges who voted in the majority, and labeling of the opinion as simply an illegitimate power grab by judges who are Democrats. House Speaker Mike Turzai, R-Allegheny, and Senate President Pro Tempore Joe Scarnati, R-Jefferson, in a joint statement issued on February 20, 2018, denounced the Pennsylvania Supreme Court’s map. After vowing to fight it in court, they asserted: ``The League of Women Voters maintains that it filed this suit in order to take partisan politics out of the Congressional redistricting process. This map illustrates that the definition of fair is simply code for a desire to elect more Democrats." Were the Republicans correct in this assertion? The evidence provided above suggests not. Given the probabilistic nature of projections, our own simulations reinforce the view that the partisan consequences of the \textit{Court Remedial} plan are within the parameters expected of a non-partisan plan. \new{As the results of the 2018 midterm elections verified, the court map demonstrated symmetry in the treatment of the two parties.}
\par
We turn next to the issue of whether the proposed Republican \textit{Joint Legislative} Plan and the map offered by the Democratic Governor could be considered partisan gerrymanders. Remarkably, this question was not really addressed by the Pennsylvania Supreme Court. Indeed, the Court’s endorsement of the plan that it ultimately adopted is couched entirely in good government terms (Opinion and Order, February 19, 2018, at pp. 6-7, with internal footnotes omitted), with no discussion of its partisan implications, or those of the other alternatives proposed.
% ================================================================= 
% -- FOOTNOTE -- FOOTNOTE -- FOOTNOTE -- FOOTNOTE -- FOOTNOTE --  %
% -----------------------------------------------------------------
    \footnote{We should also note that the Court in its adopted plan pairs two Democratic incumbents in Philadelphia (though one, Rep. Brady, was already intending to retire,), three others who are retiring (Reps. Meehan, Dent, Shuster), and the vacant incumbent seat in old district 18 with an incumbent now running for the Senate (Rep. Barletta). Its final Order does not even bother to mention the consequences for incumbents of the adopted plan. Presumably this is because the 2011 was so tainted overall by what we might call ``partisan greed" that no deference is required to the contorted lines in it that yielded those incumbencies. But, since the Court says nothing about its reasons for not mentioning incumbent effects, we can only guess.}
% ----------------------------------------------------------------- 
% -- END FOOTNOTE -- END FOOTNOTE -- END FOOTNOTE -- END FOOTNOTE %
% =================================================================
Describing the adopted plan, the Court says:
% =================================================================
% -- QUOTE -- QUOTE -- QUOTE -- QUOTE -- QUOTE -- QUOTE -- QUOTE  %
% -----------------------------------------------------------------
        \begin{quote}
            ``It is composed of congressional districts which follow the traditional redistricting criteria of compactness, contiguity, equality of population, and respect for the integrity of political subdivisions. The Remedial Plan splits only 13 counties. Of those, four counties are split into three districts and nine are split into two districts. The parties, intervenors, and amici differ in how they calculate municipal and precinct splits, and, as noted earlier, the Legislative Respondents suggest that updated data on precinct and municipal boundaries does not exist. The Remedial Plan is superior or comparable to all plans submitted by the parties, the intervenors, and amici, by whichever Census-provided definition one employs (Minor Civil Divisions, Cities, Boroughs, Townships, and Census Places)

            The compactness of the plan is superior or comparable to the other submissions, according to the Reock, Schwartzberg, Polsby-Popper, Population Polygon, and Minimum Convex Polygon measures described in the Court’s January 26 Order. Here, too, the parties, intervenors, and amici disagree on the precise ways to calculate these measures, and some failed to deliver compactness scores with their submissions. By whichever calculation methodology employed, the Remedial Plan is superior or comparable. Finally, no district has more than a one-person difference in population from any other district, and, therefore, the Remedial Plan achieves the constitutional guarantee of one person, one vote.

            It is composed of congressional districts which follow the traditional redistricting criteria of compactness, contiguity, equality of population, and respect for the integrity of political subdivisions. The Remedial Plan splits only 13 counties. Of those, four counties are split into three districts and nine are split into two districts. The parties, intervenors, and amici differ in how they calculate municipal and precinct splits, and, as noted earlier, the Legislative Respondents suggest that updated data on precinct and municipal boundaries does not exist. The Remedial Plan is superior or comparable to all plans submitted by the parties, the intervenors, and amici, by whichever Census-provided definition one employs (Minor Civil Divisions, Cities, Boroughs, Townships, and Census Places)

            Accordingly, this 19th day of February 2018, the Court orders as follows: First, the Pennsylvania primary and general elections for seats in the United States House of Representatives commencing in the year 2018 shall be conducted in accordance with the Remedial Plan\dots"
          \end{quote}
% -----------------------------------------------------------------
% -- END QUOTE -- END QUOTE -- END QUOTE -- END QUOTE -- END QUOT %
% =================================================================
\par
We believe that the evidence provided above is fully consistent with the claim by the Democratic governor that \textit{Joint Legislative} was a partisan gerrymander. Indeed, in the terminology previously defined, we would call it a \textit{stealth gerrymander}. \new{The effects of the Joint plan more closely resembled the unconstitutional plan than any of the other maps analyzed here.} The degree to which the \textit{Joint Legislative} plan is a \textit{stealth gerrymander} can be best understood by looking at the good Democratic year of 2008 in Table \ref{tab:plansummary} where, in the non-partisan plans (\textit{Court Remedial}, the first two Authors’ Plans [\textit{V1} \& \textit{V2}], \textit{DailyKos}), the Republicans averaged under 7 districts using 2008 presidential and composite election projections, but in the 2018 \textit{Joint Legislative} plan as in the \textit{2011 Enacted} plan determined to be unconstitutional, they win 11 seats. \textit{Ceteris paribus}, because of inefficiency in the distribution of Democratic voters, Republicans might pick up about a one seat advantage from ``neutral" redistricting using good government criteria so, if the state is 50-50 in partisan vote share, a ``neutral" plan might not be 9R-9D, but more like 10R-8D, but certainly not 13R-5D.
% =================================================================
    % ------- FOOTNOTE --------------------------------------------
        \footnote{\citet{Chen2013}, for example, estimate this bias as about 1.45 seats (8\%).  }
    % ------- END FOOTNOTE ----------------------------------------
% =================================================================
However, there are \underline{many} different possible \textit{neutral} plans.
% =================================================================
    % ------- FOOTNOTE --------------------------------------------
        \footnote{Indeed, many proposals were submitted to the court or released on the internet that were not taken into consideration by the court.}
    % ------- END FOOTNOTE ----------------------------------------
% =================================================================
In some reasonable proportion of the ones drawn by Professor Chen for his expert witness testimony in the case, when the state votes shares are proportional, the Democratic seat shares would be closer to 9D. In his expert witness report, Professor Chen shows in his first set of simulations using only traditional redistricting criteria that compared to the actual 13R-5D districts of the \textit{2011} plan, of the 500 simulations, \underline{none} of them resulted in a 13R-5D split. The mode was 9R-9D, while average was 8.53 seats for the Republicans.
% =================================================================
    % ------- FOOTNOTE --------------------------------------------
        \footnote{\href{https://www.pubintlaw.org/wp-content/uploads/2017/06/Expert-Report-Jowei-Chen.pdf}{Expert Witness Report, Jowei Chen} (Pg. 15, 16, Figure 2).}
    % ------- END FOOTNOTE ----------------------------------------
% =================================================================
The 11R-7D outcome in the proposed \textit{Joint Legislative} plan is beyond \new{all but 93} of the \new{1,000} outcomes Professor Chen simulated, and the 13R-5D in the \textit{2011 Enacted} plan was found to be simply outside the bounds of statistical probability. \new{The partisan outcomes can be evaluated along with adherence to traditional redistricting criteria. We determine a plan to be a \textit{stealth gerrymander} when outcomes suggests asymmetry between party's ability to elect candidates, but traditional criteria is indistinguishable from random simulations. Additionally, the asymmetry can not be attribured to geographical concentration of partisans. On page 15 of Dr. Chen's expert witness report, he shows that none of his randomly simulated plans had compactness scores as low as the \textit{2011 Enacted} plan. Additionally, none split as many municipalities, and the most amount of counties split was 19, compared to the 28 in the \textit{Enacted} plan. Alternatively, the \textit{Joint Submission} fits within the bounds of expected compactness and expected number of county splits. It also is second worse all \underline{ALL} measures of gerrymandering in Table \ref{tab:gerry}, with only the 2011 enacted plan preforming worse. It, therefore, fits our definition as a \textit{stealth gerrymander}.}
\par
However, having labeled the \textit{Joint Legislative} plan a \textit{stealth gerrymander}, it is worth reminding readers that, had this plan been the original plan adopted by Republicans, it is very much an open question as to whether it would have been rejected by the Pennsylvania Supreme Court. Even though, in partisan terms, it is virtually identical to the \textit{2011} plan, it is considerably more consistent with good government criteria (see Table \ref{tab:plansummary}) -- undoubtedly, not that much worse than the \textit{Court Remedial} plan -- that it would have required detailed political election analyses like those we have done here to demonstrate its \textit{stealth gerrymander} features.
\par
Because the \textit{Joint Legislative} plan was not passed by the State of Pennsylvania legislature, the Court felt no need for particular deference to it (or to the \textit{Governor’s} plan, for that matter.) Thus, it was not bound by the normal supposition with respect to districting that a plan authorized by the state need not be the ``best possible," but only \underline{NOT} unconstitutional. That ``deference" to legislative judgments and the criteria they reflect, might well have tipped the balance toward acceptance of the \textit{Joint Legislative} plan were it offered as having been fully sanctioned by the State of Pennsylvania (the duly elected legislature and governor).
% =================================================================
    % ------- FOOTNOTE --------------------------------------------
        \footnote{The Court picked the plan among those before it that \underline{most closely satisfied good government criteria} -- which happily, thanks to Professor Persily’s expertise as consultant to the Court, turned out to be its own plan, and thus a plan which the Court could know with certainty was not intended as a partisan gerrymander for either party.}
    % ------- END FOOTNOTE ----------------------------------------
% =================================================================
On the other hand, in the presence of this \textit{stealth gerrymander} rather than the blatant gerrymander that came before it, the Pennsylvania Court might well have relied on \underline{partisan impact} evidence to reach a conclusion of unconstitutionality. The language of the Final Order implementing the Court’s own plan strongly suggests this possibility. There, the Court said about the 2011 plan that it




\par
Finally, we consider the \textit{Governor’s} plan. That plan, unlike the \textit{Joint Legislative} plan, shows itself to be responsive to changes in voting patterns. In particular, the Democrats are projected to win three fewer seats in a good Republican year (2016) than in a good Democratic year (2008). See Table \ref{tab:plansummary} and discussion of alternative plans above. There are no clear reasons to consider this plan a partisan gerrymander. The best evidence that the Governor’s plan was \underline{not} a pro-Democratic gerrymander is evidenced by the shape of its districts.  Given the geographic concentration of Democrats in Pittsburgh and Philadelphia, had this truly been a pro-Democratic gerrymander, it would have had unnecessary splits of county and municipality boundaries in the areas of Allegheny and Philadelphia counties. These are also absent in the Court plan, along with our plans and that of the DailyKos. We have no way of knowing why the Democrats took a different tack than the Republicans in framing their remedy plan. In light of the court's remedy plan, Republicans would have been better off passing the Governor's plan. In failing to do so, the new maps will be less responsive to pro-Republican changes in state-wide vote shares.
%
%
%%%%%%%%%%%%%%%%%%%%%%%%%%%%%%%%%%%%%%%%%%%%%%%%%%%%%%%%%%%%%%%%%%%
% =================================================================
        %%%%%%%%%%%%%%%%%%%%%%%%%%%%%%%%%%%%%%%%
% =================================================================
\section{Conclusions and Lessons for the future; The Potential Impact of LWV}
% =================================================================
        %%%%%%%%%%%%%%%%%%%%%%%%%%%%%%%%%%%%%%%%
% =================================================================
%%%%%%%%%%%%%%%%%%%%%%%%%%%%%%%%%%%%%%%%%%%%%%%%%%%%%%%%%%%%%%%%%%%
%
%
While the Pennsylvania court opinion is limited to Pennsylvania, and thus it might have seemed of importance only in Pennsylvania, in footnote 71 of the Opinion (slip op. pp.116-117), the court took what we regard as a rather unusual step. It issued what can only be called an invitation to other state courts to use the same logic it used to invalidate partisan gerrymanders in their own state. As we noted earlier, the Court pointed out that there are twelve states whose constitutions contain election clauses identical to the Pennsylvania charter, requiring elections to be ``free and equal": Arizona, Arkansas, Delaware, Illinois, Indiana, Kentucky, Oklahoma, South Dakota, Oregon, Tennessee, Washington, and Wyoming.

Only a handful of these states are ripe for partisan gerrymandering challenges -- in that (ca. 2017) three lack unified party control of the state, two are single district states, and some already have a commission drawing plans and, in others, \textit{indicia} of gerrymandering are missing. We show in Table \ref{tab:compare_measures} various information that can help us form beliefs about whether or not the congressional plans in these dozen states are partisan gerrymanders. However, as pointed out in the table notes, some measures are not well suited for calculation when the seats at issue are few in number.
%
%
% =================================================================
% ------- TABLE ---------------------------------------------------
    \begin{center}\textbf{TABLE \ref{tab:compare_measures} ABOUT HERE}\end{center}
                    \input{Tables/_tab_compare_measures.tex}
% ------- END TABLE -----------------------------------------------
% =================================================================
%
%
In looking to the future, we should also note that some states that are generally regarded as among the most pernicious partisan gerrymanders, Michigan, Ohio,  North Carolina, Wisconsin are not included among the twelve, nor is Maryland.  On the other hand, we should also note that a ``free and equal" elections clause is not the only avenue state courts might use to attack partisan gerrymanders in the future. As University of Kentucky College of Law Professor Joshua Douglas has pointed out, virtually every state constitution protects voting rights more explicitly than the U.S. Constitution does. In addition to the thirteen states that require elections to be ``free and equal", an additional thirteen have state constitutional provisions that require elections to be ``free and open", and this clause could, in principle, be used in exactly the same way as the ``free and equal" clause.
% =================================================================
    % ------- FOOTNOTE --------------------------------------------
        \footnote{We are indebted to Jonathan Lai of the Philadelphia Inquirer (personal communication, April 2018) for calling this information to our attention. See a more detailed discussion in \citet{Elmendorf2018}.}
    % ------- END FOOTNOTE ----------------------------------------
% =================================================================
Such a clause is found in a number of the states widely regarded as having the worst congressional gerrymandering. \new{Additionally, other routes besides the courts are recognized as potential solutions to egregious gerrymandering \citep{Wang_et_al_2019_Labortories_UPJCL}.} We believe that the \textit{LWV} decision and the further comparative analyses of alternative plans presented in the previous section of this paper will help other state courts navigate their way to decisions that strike down some partisan gerrymanders as unconstitutional, while allowing others to remain in place on the grounds that either they are not that severe, or they are unlikely to be lasting in that they have sufficiently many competitive seats as to be responsive to changes in voting patterns.




	To simplify the exposition, we report the Republican share of the two-party vote ($\delta_{i}$). Districting plans are represented by $\mathcal D $, (e.g., $\mathcal D_{enacted}$, $\mathcal D_{remedial}$, \dots, $\mathcal D_{j}$), and each election in year $ y $ has a district vote distribution $ \Delta_{y} = [\delta_{y1}, \delta_{y2}, \dots, \delta_{yi}] $. To find the overall state-wide vote, we calculate the average district vote share, $ \bar{\Delta}_{\mathcal D y} = \frac{1}{n}\sum\limits_{i=1}^{n} \delta_{yi} $. By averaging, we reduce the influence of turnout variation between districts,. This average is a useful state-wide estimate of voter sentiment \citep[see e.g.][]{Kastellec_et_al_2008_PS}.
	
	